\documentclass[12pt, letterpaper]{article}
\usepackage{listings}
\title{Comp Sci}
\author{Jackson Smith}
\date{May 2024}

\begin{document}
\maketitle

\section{Data Structures}

\subsection{Array}

\subsection{Tree}
\subsubsection{Binary Tree}
\subsubsection{Binary Search Tree}
\subsubsection{Complete Binary Tree}

\subsection{Graph} 
\subsubsection{Directed Graph} 

\section{Asymtotic Notation} 

\section{Common Algorithms}

\subsection{Sorting}

\subsubsection{Merge Sort}

\paragraph{Description}
Sorts a subarray A[p:r] starting with the entire array A[1:n] and recursing 
down to smaller and smaller subarrays. 
\begin{enumerate}
	\item Divide the subarray A[p:r] to be sorted into two adjacent subarrays, 
		each of half their size. To do so, compute the midpoint q of A[p:r]
		(taking the average of p and r), and divide A[p:r] into subarrays A[p:q]
		and A[q + 1: r]. 
	\item Conquer by sorting each of the two subarrays A[p:q] and A[q + 1:r]
		recursively using merge sort. 
	\item Combine by merging the two sorted subarrays A[p:q] and A[q + 1:r] back
		into A[p:r], producing the sorted answer. 
\end{enumerate}

\paragraph{Implementation}
\begin{lstlisting}[language=C] 
	int get_midpoint(int start, int end) {
		return floor(start+end)/2; 
	}
\end{lstlisting}

\end{document}
